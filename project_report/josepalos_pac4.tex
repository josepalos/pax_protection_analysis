\documentclass{article}
\usepackage[utf8]{inputenc}
\usepackage{graphicx}
\usepackage{caption}
\usepackage{subcaption}
\usepackage{listings}
\usepackage{xcolor}
\usepackage{placeins}
\usepackage{fancyhdr}
\usepackage{hyperref}

\newcommand\subject{Visualització de dades}
\newcommand\activity{PACX}

\title{\large \subject \\ \activity}
\author{Josep Alòs Pascual}
\date{\today}
\makeatletter

\pagestyle{fancy}
\lhead{\subject}
\rhead{\activity}
\rfoot{Josep Alòs Pascual}

\begin{document}
\maketitle

\section{Introducció}
Aquest document presenta el projecte final de visualització de dades sobre el
dataset d'acords de pau. S'explicarà com s'ha implementat la visualització

A continuació es presenta l'estructura del document
\tableofcontents

\subsection{Resum del projecte}
Per determinar com es protegeixen els grups socials d’un territori després d’un
conflicte, en aquest projecte s’estudia com diferents acords de pau d’entre el
1990 i 2016 determinen mesures de protecció per aquests grups. En aquest
document es presenten les dades i una visualització que serveix per reflectir
les conclusions d’aquest estudi de forma senzilla d’entendre però a la vegada
útil.

\subsection{Detalls de la visualització}
\begin{itemize}
    \item \textbf{Títol de la visualització}: Protecció dels diferents grups
        socials en acords de pau
    \item \textbf{URL de la visualització}: \url{https://pax-protection-analysis.herokuapp.com/}
    \item \textbf{URL del codi font}: \url{https://github.com/josepalos/pax_protection_analysis}
    \item \textbf{Llicència del projecte}: \textit{Creative Commons Attribution-NonComercial-ShareAlike 4.0
        Internacional}\footnote{\url{https://creativecommons.org/licenses/by-nc-sa/4.0/}}
    \item \textbf{Dades utilitzades}: Conjunt de dades sobre 1.500 acords de pau
        des del 1990 fins l'actualitat; recopilats
        per la universitat d'Edinburg. Pot ser obtingut de la seva pàgina
        \url{https://www.peaceagreements.org}, i es troba llicenciat sota les
        la mateixa llicència que aquest projecte.
\end{itemize}

\section{Compliment dels objectius inicials}
Inicialment es va proposar que aquest projecte complís amb els següents
objectius:
\begin{enumerate}
    \item Oferir una visualització interactiva sobre les dades dels acords de
        pau mitjançant l'eina \textbf{Tableau}.
    \item Preprocessar les dades rebudes, eliminant dades no utilitzades i/o
        convertint les variables que s'utilitzen al format adequat; utilitzant
        \textbf{Python}.
    \item Centrar la visualització a l'estudi de les proteccions que reben els
        diferents grups d'una societat després d'un conflicte.
    \item Avançar des d'una visió general de les dades cap a visualitzacions
        específiques sobre el tema d'interès.
\end{enumerate}

El primer objectiu s'ha complert, si bé no de la forma esperada. S'ha optat per
canviar l'eina que genera la visualització de \textbf{Tableau} a \textbf{D3}.
Això s'ha fet ja que, un cop el desenvolupador s'ha familiaritzat amb els
conceptes bàsics d'aquesta nova llibreria, dóna més flexibilitat a l'hora de
dissenyar i implementar les visualitzacions.

El segon objectiu s'ha complert, però altre cop canviant l'eina utilitzada. S'ha
optat per dur a terme el preprocessat de les dades mitjançant \textbf{JavaScript}
en el moment de carregar el conjunt de dades a l'aplicació. Aquest canvi s'ha
fet ja que, un cop no tenim la restricció de \textbf{Tableau} que no es poden
preprocessar les dades amb la facilitat que dóna un \textit{script} escrit
específicament per aixó, a la vegada que es fa servir \textbf{JavaScript} per
implementar l'aplicació, no cal afegir un pas que s'ha d'executar a priori de
desplegar la visualització.

Per evaluar els dos últims punts, s'ha d'especificar quines visualitzacions
s'havien previst inicialment:

\begin{enumerate}
    \item Visió general de les dades
        \begin{enumerate}
            \item Evolució de la quantitat d'acords durant els anys
            \item Mapa mundial
        \end{enumerate}
    \item Visió focalitzada en mostrar la distribució dels acords segons quin
        nivell de protecció ofereixen als diferents grups.
        \begin{enumerate}
            \item Quantitat d'acords que estableixen alguna protecció per algun
                grup cada any.
            \item Quantitat de grups protegits segons el nivell i acords
        \end{enumerate}
    \item Relacions interessants entre variables dels acords.
        \begin{enumerate}
            \item Relació entre els tipus d'acords i la protecció dels grups
            \item Relació entre la protecció dels diferents grups
            \item Relació entre la protecció dels grups i mencions al
                \textit{National Human Rights Institution (NHRI)}
        \end{enumerate}
\end{enumerate}

S'han desenvolupat totes les visualitzacions previstes amb l'estructura
especificada, per tant també s'ha complert els dos últims objectius. L'última
visualització, però, s'ha canviat d'un núvol de punts a un gràfic amb cercles
de diferents àrees segons el valor que es vol mostrar, per ser consistent amb
el gràfic que es té al costat. A més, en aquests dos gràfics s'ha afegit
l'opció de canviar quines variables es comparen mitjançant un selector sota
del gràfic.

Referent a la utilitat de la visualització per mostrar el que es vol, considero
que en general si que és una eina bona ja que s'ha seguit el disseny proposat a
la pràctica 3, i s'ha comprovat que ens permet evaluar de forma ràpida les
relacions entre les variables. Tot i això, els dos últims gràfics no aporten
tanta informació com s'esperava, degut a que el pes dels acords que no
protegeixen els grups (en el gràfic corresponen al 0, 0) és considerablement
superior als altres acords, i no s'aconsegueix veure amb claredat les relacions
de protecció.


\section{Que s'ha après en el projecte}
Durant el desenvolupament d'aquest projecte he après el procés a seguir durant
el disseny i implementació d'un projecte de visualització (l'anàlisi de les
dades, proposta de disseny, implementació\dots). A més a més, he après a fer
servir eines per implementar visualitzacions de dades: \textbf{Tableau} en les
primeres entregues de les pràctiques, i \textbf{D3} en la darrera entrega. Si
bé no he explorat molt l'eina \textbf{Tableau}, de \textbf{D3} he après molt
sobre aquesta llibreria, com per exemple com gestiona l'actualització dels
elements de la pàgina un cop canvien les dades. A més, m'ha permès integrar-ho
amb \textbf{React}, un \textit{framework} que estava aprenent pel meu compte
paral·lelament, i per tant m'ha servit per practicar aquesta altra eina. Per
últim, també comentar que m'ha semblat molt interessant les dades que s'han
fet servir, i si bé ara mateix no en tinc cap ús pràctic a part d'ampliar el
coneixement, en un futur és possible que torni a utilitzar les dades descobertes.

\section{Descripció tècnica del projecte}
En aquest apartat s'explicarà com s'ha dut a terme la implementació de l'aplicació.

\subsection{Preprocessat de les dades}
Primer de tot s'ha d'eliminar les columnes de dades que no necessitem per l'anàlisi.
Els atributs que s'han conservat són:
\begin{itemize}
    \item Con: País (o països) involucrats en l'acord de pau
    \item Contp: Tipus de conflicte
    \item AgtId: Identificador del tractat
    \item Agt: Nom del tractat
    \item Dat: Data del tractat. Només s'ha conservat l'any
    \item GCh: Protecció dels infants
    \item GDis: Protecció de persones discapacitades
    \item GAge: Protecció de la tercera edat
    \item GMig: Protecció de treballadors migrants
    \item GRa: Protecció de grups racials, ètnics, o nacionals
    \item GRe: Protecció de grups religiosos
    \item GInd: Protecció de grups indígenes
    \item GRef: Protecció de refugiats
    \item GSoc: Protecció de classes socials
    \item GOth: Protecció d'altres grups
    \item HrNi: Provisió d'un cos que controli els drets humans.
\end{itemize}

Per cada variable de protecció s'ha conservat també les variables de cada nivell.
Aquestes són ``nom de grup''+{Rhet,Antid,Subs,Other}.

Per últim, s'ha precalculat un agregat de cada tractat de pau, que consisteix en el
recompte de grups que estan protegits per cada nivell, obtenint així tres variables
addicionals
\begin{itemize}
    \item rhetoricalCount
    \item antiDiscriminiationCount
    \item substantiveCount
    \item otherProtectionsCount
\end{itemize}

\subsection{Dades addicionals utilitzades}
Per generar el mapa s'ha utilitzat les dades proporcionades per \textbf{TopoJSON}
corresponents a l'àtlas mundial. Es poden obtenir des de la pàgina del projecte
(\url{https://github.com/topojson/world-atlas}) o des de la pàgina
\textit{UNPKG}\footnote{UNPKG és una pàgina que actua com a proveïdor de continguts
per \textbf{NPM}}. S'ha preprocessat aquestes dades per tal de tenir un mapeig
entre els identificadors del país i el seu nom. A més, s'ha especificat un mapeig
addicional per a que els noms dels països coincideixin amb els del conjunt de dades
dels acords de pau.

\subsection{Eines utilitzades}
El projecte s'ha implementat utilitzant el \textit{framework} \textbf{React} per
tal de gestionar l'estructura de l'aplicació. Aquesta eina permet partir
l'aplicació en components reutilitzables per tal de garantir les bones pràctiques
de disseny.

Dins d'aquest \textit{framework} s'han creat components per cada una de les
visualitzacions. Totes les visualitzacions menys dues s'han implementat
utilitzant un element \textit{SVG} controlat per la llibreria \textbf{D3}. Les
altres dues s'han implementat utilitzant una taula, generada de forma dinàmica
amb \textbf{React}. Per genera el mapa del món s'ha utilitzat \textbf{D3}
juntament amb la llibreria \textbf{TopoJSON}\footnote{\url{https://github.com/topojson/topojson}}.

La càrrega de dades es fa de forma asíncrona un cop es carrega l'aplicació, i
les dades es guarden al seu estat. Això fa que, un cop s'han carregat,
\textbf{React} actualitza els components que la fan servir, i aquests components
són els encarregats de passar aquestes dades a la llibreria de visualitzacions.


\section{Passos futurs i punts a millorar}
A continuació s'esmenten possibles canvis i millores a implementar en futures
iteracions del projecte:

\begin{itemize}
    \item Utilitzar \textbf{Bootstrap} per fer la pàgina \textit{responsive}.
    \item Millorar els processos de càrrega, parseig, i mostra de les dades.
        Actualment no suposen cap problema ja que el volum de dades usades és
        petit, però personalment considero que el codi pot millorar un cop
        tingui més domini de \textbf{React} i \textbf{D3}.
    \item Afegir filtres a les visualitzacions (per exemple, a la quarta
        visualització afegir la possibilitat de filtrar per anys).
    \item Afegir més informació als \textit{tooltips} que apareixen en situar
        el ratolí sobre els diferents gràfics.
    \item Afegir textos explicatius a cada gràfic, ja sigui amb una capa
        desplegable o sota d'aquests.
    \item Afegir transicions animades quan s'actualitzen els gràfics.
\end{itemize}

\end{document}
